\section{Основные классы биологический молекул, ДНК, генетический код}
Кто хочет подробней открывайте Альбертса с 59ой страницы. Тут только малая часть того что там есть
\subsection{Основыне классы биологический молекул}
\paragraph{Углеводны} Они же сахара. Содержат альдегидную группу $\ce{C=O}$ А так же несколько гидроксильных групп $\ce{O - H}$. Простейшим примером является глюкоза $\ce{C6H12O6}$, которая частный случай моносахарида $\ce{(CH2O)_n}$ где $n$ любое натуральное число. Углеводы могут существовать либо в форме кольца, либо
в виде открытой цепи. К одному кольцо может через атом углерода альдегидной группы присоединиться еще одно кольцо, образуя дисахарид. Например мальтоза (см картинку ниже). Можно образовывать еще большие последовательности, называются полисахаридами.
\begin{figure}[H]
	\includegraphics[scale = 0.2]{2glu}\\
	\includegraphics[scale = 0.2]{2mal}
	\caption{Глюкоза и мальтоза}
\end{figure}
Глюкоза служит главным источником энергии во многих клетках. В результате последовательного ряда реакций окисления эта гексоза превращается в различные производные Сахаров с меньшей длиной цепи и в конечном итоге распадается до $\ce{CO2}$ и $\ce{H2O}$. В ходе распада глюкозы высвобождается энергия и генерируется восстановительная способность, без чего невозможно протекание
биосинтетических реакций. Высвобождающаяся энергия и генерируемая восстановительная сила запасаются в форме двух важнейших соединений -
АТР и NADH. Так же из углеводов состоит важный внеклеточный структурный материал (например целлюлоза) 
\paragraph{Липиды} они же жиры. У них обычно имеются две различные части: длинная углеводородная цепь,
которая имеет гидрофобный характер (водонерастворима) и химически мало активна, и карбоксильная группа, ионизирующаяся в растворе, крайне
гидрофильная (водорастворимая) и легко образующая эфиры и амиды.\\
Жирные кислоты являются ценным источником энергии, поскольку их
расщепление сопровождается образованием такого количества АТР, которое в два раза
превышает образование АТР при расщеплении такого же количества (по массе)
глюкозы. Жирные кислоты запасаются в цитоплазме многих клеток в виде капелек
триацилглицеролов (триглицеридов). Молекулы триацилглицеролов состоят из трех
цепей жирных кислот, каждая из которых присоединена к молекуле глицерола  именно так устроены животные жиры, с которыми мы имеем дело в
повседневной жизни.\\
Но самая важная функция жирных кислот - участие в построении клеточных
мембран. Эти тонкие плотные пленки, которыми одеты все клетки и внутриклеточные
органеллы, состоят главным образом из фосфолипидов
\begin{figure}[H]
	\centering
	\includegraphics[scale=0.3]{2lipid}
\end{figure}
\paragraph{Аминокислоты} органические вещества которые состоят из аминогруппы и карбоксильной круппы
\begin{figure}[H]
	\centering
	\includegraphics[scale= 0.7]{2amin}
\end{figure}
Аминокислоты служат строительными блоками при синтезе белков
- длинных линейных полимеров аминокислот, соединенных «хвост к голове» при
помощи пептидной связи между карбоксильной группой одной аминокислоты и
аминогруппой другой. В белках встречается обычно 20 аминокислот с
разными радикалами. Одни и те
же 20 аминокислот неоднократно повторяются во всех белках, в том числе в белках
бактериального, животного и растительного происхождения.
\paragraph{Нуклеотиды} Нуклеотиды являются сложными эфирами нуклеозидов и фосфорных кислот.
\begin{figure}[H]
	\includegraphics[scale = 0.7]{2nucle}
\end{figure}
Цитозин (С), тимин (Т) и урацил (U)
называются пиримидиновыми основаниями, так как они представляют собой простые производные шестичленного пиримидинового кольца;
гуанин (G) и аденин (А) - пуриновые основания, второе пятичленное кольцо которых сконденсировано с шестичленным циклом\\
Нуклеотиды могут выступать в качестве переносчиков энергии. При этом трифосфатный эфир аденина АТР (рис. 2-9) гораздо чаще, чем
другие нуклеотиды, участвует в переносе энергии между сотнями индивидуальных внутриклеточных реакций. Энергия высвобождающаяся в результате гидролиза АТФ может использоваться в любой другой реакции которая проходит с поглощением энергии.
\begin{figure}[H]
	\includegraphics[scale=0.4]{2atf}
\end{figure}
Другие производные
нуклеотидов служат переносчиками отдельных химических групп, таких, как атомы водорода или остатки Сахаров, с одной молекулы на другую. Кроме того, циклическое фосфорилированное производное аденинациклический AMP (cAMP) -служит универсальным внутриклеточным сигналом и
регулирует скорость множества различных внутриклеточных реакций. \\

Нуклеотиды служат
строительными блоками для синтеза нуклеиновых кислот - длинных полимеров, в которых нуклеотидные субъединицы соединяются между собой
ковалентной связью, образуя фосфорный эфир между 3'-гидроксильной группой остатка сахара одного нуклеотида и 5'-фосфатной группой. Нуклеиновые кислоты, сахар которых представлен рибозой, называются рибонуклеиновыми кислотами или \textbf{РНК}; они
содержат основания A, U, G и С. Те нуклеиновые кислоты, в состав которых входит дезоксирибоза (в ней гидроксильная группа при С-2 рибозы
замещена на атом водорода), называются дезоксирибонуклеиновыми кислотами или \textbf{ДНК}; они содержат основания А, Т, G и С. Способность к спариванию G c C и A c T лежит в основе механизмов хранения и передачи наследственно информации.

\subsection{Формы ДНК}
Наиболее распространой является В-форма. В этой форме находится основная часть ДНК в клетках. При такой организации плоскости азотистых оснований практически перпендикулярны оси двойной спирали, и каждая пара повёрнута относительно предыдущих на $36^\circ$. На один виток спирали приходится примерно 10 нуклеотидных пар (9,7 и 10,6 в различных кристаллах)(2), а длина составляет 3,4 нм.
\begin{figure}[H]
	\centering
	\includegraphics[scale = 0.5]{2adnk}
\end{figure}
Существенным отличием А-формы от В-формы является то, что в А-форме пары оснований сдвинуты к периферии спирали почти на половину её радиуса, в результате чего пространство вдоль оси оказывается пустым. Большая бороздка при этом становится глубже и уже, а малая бороздка оказывается шире и более плоской
\begin{figure}[H]
	\includegraphics[scale=0.6]{2zdnk}
\end{figure} 
Z форма представляет собой левозакрученную спираль с длиной витка 4,4 нм, на который приходится 12 нуклеотидных пар.
\subsection{Отличие эукариот от прокариот}
Прокариоты не имеют оформленного ядра. Их хромосомы имеют кольцевую форму. А гены объеденены в опероны.
Эукариотические клетки по определению и в отличие от прокариотических имеют ядро (по гречески «карион»). Ядро, в котором
находится большая часть клеточной ДНК, ограничено двойной мембраной. 

\subsection{Генетический код}
\textbf{Генетичесий код - совокупность правил, согласно которым в живых клетках последовательность нуклеотидов (ген и мРНК) переводится в последовательность аминокислот (белок).}\\
Ген - участок днк определяющий определенную последовательность аминокислот.\\
Аллели - различные формы одного и того же гена, расположенные в одинаковых участках (локусах) гомологичных хромосом, определяют направление развития конкретного признака

\subsection{Митоз и мейоз}
Существует два типа деления клеток эукариот: митоз и мейоз. При митозе хромосомный набор (плоидность) клетки не меняется, обе дочерние клетки полностью генетически идентичны исходной. Это обычный способ деления клеток, например, при формировании тел многоклеточных животных и растений. В результате мейоза, который включает в себя 2 деления, из одной диплоидной клетки получается 4 гаплоидных, причем все они генетически отличаются друг от друга. Это может происходить при формировании гамет или спор.\\

\textbf{Мейоз} (редукционное деление клетки) — деление, в процессе которого из одной диплоидной клетки получаются 4 гаплоидные клетки.
\begin{figure}[H]
	\includegraphics[scale=0.5]{2mi}
\end{figure}
Мейоз у животных наблюдается при формировании гамет (гаметогенезе). Мейоз у растений и грибов, как правило, происходит при образовании гаплоидных спор. У различных одноклеточных эукариот мейоз может наблюдаться на разных стадиях жизненного цикла. Для восстановления диплоидности в цикле всегда необходимо слияние гаплоидных клеток (оплодотворение).\\
Мейоз состоит из двух делений. Первое из них является собственно редукционным, то есть именно в ходе первого деления уменьшается плоидность клетки. Причиной этого служит расхождение гомологичных хромосом («материнской» и «отцовской») по двум разным дочерним клеткам. Второе деление аналогично митозу и называется эквационным (то есть «равным»). Плоидность в результате второго деления не меняется. В ходе этого деления, как и при митозе, расходятся сестринские хроматиды (копии ДНК). Между двумя делениями мейоза отсутствует репликация ДНК (так как «цель» мейоза — уменьшить плоидность клетки, увеличивать количество ДНК здесь незачем).\\
В профазе I деления мейоза происходит важнейший процесс, относящийся к генетической рекомбинации — кроссинговер, то есть обмен участками гомологичных хромосом.\\

\textbf{Митоз} - непрямое деление клетки, наиболее распространённый способ репродукции эукариотических клеток. Биологическое значение митоза состоит в строго одинаковом распределении хромосом между дочерними ядрами, что обеспечивает образование генетически идентичных дочерних клеток и сохраняет преемственность в ряду клеточных поколений. \textbf{Митоз лежит в основе бесполого размножения, от отличие от мейоза.}
\begin{figure}[H]
	\centering
	\includegraphics[scale = 0.5]{2m1}
	\includegraphics[scale = 0.5]{2m2}
	\includegraphics[scale = 0.5]{2m3}
	\includegraphics[scale = 0.5]{2m4}
\end{figure}
Профаза это подготовительный этап. ДНК удваивается и расходятся, они остаются соединены только в центромере. Образуется веретено деления. Микротрубочки веретена деления прикрепляются к хромосомам.\\
Затем наступает метафаза. За счет изменения длины нитей веретена хромосомы перемещаются в среднюю часть клетки, образуя экватор деления.\\
За метафазой наступает анафаза (см. рис. 6). Центромеры сестринских хроматид разделяются, нити веретена укорачиваются, в результате дочерние хроматиды расходятся к противоположным полюсам\\
Митоз завершается телофазой, в которой восстанавливается исходная структура ядер. Вокруг каждого набора хромосом у полюсов деления формируется новая ядерная оболочка